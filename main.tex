\documentclass[12pt,a4paper,titlepage]{article}
\usepackage[utf8]{inputenc}
\usepackage[german]{babel}
\usepackage{amsmath}
\usepackage{amsfonts}
\usepackage{amssymb}
\usepackage[left=2cm,right=3cm,top=2cm,bottom=2cm]{geometry}
\usepackage{exscale}
\title{Physik-Lernzettel}
\author{Lars Funke et al.}
\date{Stand: \today}
\begin{document}
	\maketitle
	\tableofcontents
	\pagebreak
	\section{Schwingungen}
		\subsection{Erzwungene Schwingungen}
			\begin{itemize}
				\item Amplitude zeigt Schwingungsenergie
				\item Entsteht unter Einfluss einer äußeren, zeitlich periodischen Kraft
				\item Periodische Energiezufuhr 
				\item Wenn {\boldmath $ f_{E} << f_{O}$} : $\hat{s}_{\mathrm{Oszillator}}=\hat{s}_{\mathrm{Erreger}} $, Keine Phasenverschiebung 
				\item Wenn  {\boldmath $f_{E} >> f_{O} $}: Oszillator und Erreger schwingen im Gegentakt, Phasenverschiebung von $\pi$, $\hat{s}$ ist klein
				\item Wenn {\boldmath $f_{E} = f_{O} $}:Phasenverschiebung von $\frac{\pi}{2}$, $\hat{s}$ wächst endlos(Wenn Oszillator ungedämpft), \textbf{Resonanz}
				
			\end{itemize}
	\section{Wellen}
		\subsection{Eigenschaften von Wellen}
			\begin{itemize}
				\item Ausbreitung einer Störung
				\item Viele gekoppelte Oszillatoren
				\item Zeitliche ($T$) und räumliche ($\lambda$) Periodizität
				\item Punkte gleicher Phase bilden eine Wellenfront
				\item Wellennormale gibt Ausbreitungsrichtung an
			\end{itemize} 
		\subsection{Wellengleichung}
			An der Stelle $x = 0$ gilt:  
			$$ y(0,t) = \hat{y} \cdot sin(\omega t)$$
			Für eine beliebige Stelle $x$ gilt:  
			$$ y(x,t) = \hat{y} \cdot sin \left( 2 \pi \left( \frac{t}{T} - \frac{x}{\lambda} \right) \right) $$
		\subsection{Schallwellen}
			\begin{itemize}
				\item Ausbreitungsgeschwindigkeit $c$ vom Medium und der Temperatur abhängig
				\item Longitudinalwellen
				\item breiten sich dreidimensional aus
			\end{itemize}
			\begin{gather*}
				c_{\mathrm{Luft}} \approx 340 \frac{m}{s} (\mathrm{bei}\ T = 20^{\circ}\mathrm{C})\\
				c_{\mathrm{Luft}} \approx 332 \frac{m}{s} (\mathrm{bei}\ T = 0^{\circ}\mathrm{C})\\
				c_{\mathrm{Wasser}} \approx 1485 \frac{m}{s} (\mathrm{bei}\ T = 20^{\circ}\mathrm{C})\\
				c_{\mathrm{Wasser}} \approx 1404 \frac{m}{s} (\mathrm{bei}\ T = 0^{\circ}\mathrm{C})
			\end{gather*}
		\subsection{Dopplereffekt}
			\subsubsection{Feststehender Beobachter, bewegliche Schallquelle}
				Annäherung: 
				$$f' = \frac{c}{c-v}\cdot f$$
				Entfernung: 
				$$f'' = \frac{c}{c+v}\cdot f$$
			\subsubsection{Beweglicher Beobachter, feststehende Schallquelle}
				Annäherung: 
				$$f' = \frac{c+v}{c}\cdot f$$
				Entfernung: 
				$$f'' = \frac{c-v}{c}\cdot f$$
		\subsection{Huygensches Prinzip}
			Jeder Punkt einer Wellenfront kann als Ausgangspunkt einer Elementarwelle (Kreis- bzw. Kugelwelle) angesehen werden, die sich mit gleicher Phasengeschwindigkeit und Frequenz wie die ursprüngliche Welle ausbreitet.
		\subsection{Interferenz}
			\begin{itemize}
				\item Überlagerung (gleichfrequenter) Wellen
				\item Auslenkung entsprecht vektorieller Summe der Einzelwellen
			\end{itemize}
			\subsection{Berechnung konstruktive/destruktive Interferenz}
			$s:$ Abstand des betrachteten Punktes zum Erregerzentrum \\
			$\Delta s = |s_1 - s_2|$ (Gangunterschied)\\
			Wenn $\Delta s = n \lambda, n \in \mathbb{N}$: konstruktive Interferenz\\
			Wenn $\Delta s = n \lambda + \frac{\lambda}{2} = (n+\frac{1}{2}) \lambda, n \in \mathbb{N}$: destruktive Interferenz\\\\
			Phasenunterschied zweier Wellen mit Gangunterschied $\Delta s$:\\
			$$\phi = \frac{\Delta s}{\lambda} \cdot 2 \pi$$
	\pagebreak
	\section{Elektrisches Feld}
		\subsection{Allgemein}
			\glqq Im Raum um einen ruhenden elektrisch geladenen Körper wirken auf andere (geladene) Körper Kräfte. Diesen Raum nennt man elektrisches Feld."
		\subsection{Ladung}
			$$Q = I \cdot t \quad (I = const.)$$ \\
			$$Q = \int_{t0}^t I(t) \mathrm{d}\mathrm{t}$$ 
		\subsection{Feldlinien}
			\begin{itemize}
				\item Linien zwischen geladenen Körpern
				\item Beschreiben den potentiellen Weg eines Probekörpers
				\item Stehen senkrecht auf Leiteroberflächen
				\item Schneiden sich nicht, da Felder sich (vektoriell) addieren
			\end{itemize}
		\subsection{Elektrische Feldstärke}
			Die im Feld wirkende Kraft ist proportional zur Ladung des Probekörpers: 	
			$F \sim q$\\
			Die Feldstärke wird also als
			$$\vec{E} = \frac{1}{q} \cdot \vec{F} \qquad [E] = 1 \frac{N}{C}$$
			definiert. \\ \\
			Im homogenen Feld gilt: \\
			$$E = \frac{U}{d}$$
		\subsection{Coulomb-Kraft}
			Im radialsymmetrischen Feld gilt:
			$$F = \frac{1}{4 \pi \epsilon_0} \cdot \frac{q_1 \cdot q_2}{r^2}$$
		\subsection{Flächenladungsdichte}
			\begin{gather*}
				\sigma = \frac{Q}{A} \qquad [\sigma] = 1 \frac{E}{m^2} \\
				\sigma = \varepsilon_0 \cdot E \\
				\varepsilon_0 \approx 8,85419 \cdot 10^{-12} \frac{C^2}{Nm^2}
			\end{gather*}
		\subsection{Energie und Spannung}
			\subsubsection{Allgemein}			
			\begin{itemize}
				\item Auf Flächen senkrecht zu den Feldlinien ändert sich das elektrische Potential nicht (Äquipotentialflächen)
				\item Jeder geladene Körper in einem elektrischen Feld hat eine potentielle Energie
			\end{itemize}
			\subsubsection{Potentielle Energie}
				Homogenes Feld: \\
				$$E_{pot} = F \cdot \Delta s = Q \cdot E \cdot \Delta s$$ \\
				Radialsymmetrisches Feld: \\
				\begin{gather*}
					F = \frac{1}{4\pi\varepsilon_0} \cdot \frac{q_1 \cdot q_2}{r^2} \\
					\left| \Delta E_{AB} \right| = \left| \frac{q_1 \cdot q_2}{4\pi\varepsilon_0} \cdot (\frac{1}{r_A}-\frac{1}{r_B}) \right|
				\end{gather*}
			\subsubsection{Elektrisches Potential}
				$$\varphi_P = \frac{E_{pot}}{Q}$$
			\subsubsection{Elektrische Spannung}
				$$U = \frac{\Delta E_{AB}}{Q} \qquad [U] = V$$
				Für ein homogenes Feld gilt dementsprechend: \\
				$$U = \frac{Q \cdot E \cdot d}{Q} = E \cdot d$$
				Spannung entspricht der Potentialdifferenz: \\
				$$U = \varphi_B - \varphi_A$$
			\subsubsection{Elektrische Energie}
				$$\Delta E_{el.} = U \cdot Q$$
	\section{Gedämpfte Schwingungen}
		\subsection{k bestimmen}
			\subsubsection{1. Variante}
				\begin{itemize}
					\item Wenn $I_0$  bestimmmt, lies I(t) zu einem beliebigen Zeitpunkt ab, und stelle nach k um. 
					DGL aufstellen, mithilfe eines Spannungansatzes:
					 $$ U_{C}+U_{L}+U_{R}=0 $$
					\item Lösung des DGL ergibt:
					$ \omega= \sqrt{\omega_{0}^2-k^2} $ wobei $ \omega_0 = \frac{1}{\sqrt{LC}}$
					\item berechne R mit $ k = \frac{R}{2L} $
					\end{itemize}
			\subsubsection{2. Variante}
				\begin{itemize}
					\item Wenn $I_0$ nicht bekannt, lies die Stromstärke zweier benachbarter Maxima ab
					$$ \frac{I(\frac{T}{4})}{I (T+\frac{T}{4})} = \frac{I_0\cdot\sin({\omega-\frac{T}{4}})\cdot e^{-k\cdot\frac{T}{4}}}{I_0\cdot\sin(\omega \cdot ( T+\frac{T}{4}))\cdot e^{-k(T+\frac{T}{4})}} $$
					$$ \frac{e^{-k \cdot \frac{T}{4}}} {e^{-k(T+ \frac{T}{4})}} = e^{-k \frac{T}{4}+k(T+\frac{T}{4})} $$
					$$ = \frac{e^{-k\frac{T}{4}+k \cdot T+ k \frac{T}{4}}}{e^{kt}} $$
					$$ \rightarrow \ln (\frac{I (\frac{T}{4})}{I(\frac{T}{4}+I)} = k \cdot T $$
					\Large $$  \Rightarrow k= \frac{1}{T} \cdot \ln(\frac{I(\frac{T}{4})}{I(\frac{T}{4}+T)}) $$  
					\normalsize $$ \omega^2 = \frac{1}{LC} - k^2  $$
					\Large $$ L = \frac{1}{L\cdot (\omega^2+k^2)}$$
				\end{itemize}
			\subsubsection{3. Variante}
				\begin{itemize}
					\item Zeichne die Einhüllende bzw. das Zeit-Amplituden-Diagramm
					\item Lies die "Halbwertszeit" $t_\frac{1}{2}$ ab 
					\item k ermitteln : $\frac{y_0}{2} = y_0 \cdot e^{-kt_\frac{1}{2}} \leftrightarrow \frac{1}{2}= e^{-kt_\frac{1}{2}} $$ 
					\rightarrow $ \ln(\frac{1}{2}) = -k \cdot t_\frac{1}{2} $ \\
					\large $$ \Rightarrow  k= \frac{\ln(2)}{t_\frac{1}{2}} $$
				\end{itemize}
			\subsubsection{4. Variante}
				FIX ME 
				
		
\end{document}