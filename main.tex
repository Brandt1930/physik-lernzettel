\documentclass[12pt,a4paper,titlepage]{article}
\usepackage[utf8]{inputenc}
\usepackage[german]{babel}
\usepackage{amsmath}
\usepackage{amsfonts}
\usepackage{amssymb}
\usepackage[left=3cm,right=2cm,top=2cm,bottom=2cm]{geometry}
\title{Physik-Lernzettel}
\author{Lars Funke et al.}
\date{Stand: \today}
\begin{document}
	\maketitle
	\tableofcontents
	\pagebreak
	\section{Schwingungen}
	\section{Wellen}
		\subsection{Eigenschaften von Wellen}
			\begin{itemize}
				\item Ausbreitung einer Störung
				\item Viele gekoppelte Oszillatoren
				\item Zeitliche ($T$) und räumliche ($\lambda$) Periodizität
				\item Punkte gleiche Phase bilden Wellenfront
				\item Wellennormale gibt Ausbreitungsrichtung an
			\end{itemize}
		\subsection{Wellengleichung}
			An der Stelle $x = 0$ gilt:\\
			$ y(0,t) = \hat{y} \cdot sin(\omega t)$ \\\\
			Für eine beliebige Stelle $x$ gilt:\\
			$ y(x,t) = \hat{y} \cdot sin(2 \pi (\frac{t}{T} - \frac{x}{\lambda} )) $
		\subsection{Schallwellen}
			\begin{itemize}
				\item Ausbreitungsgeschwindigtkeit $c$ vom Medium und der Temperatur abhängig
				\item Longitudinalwellen
				\item breiten sich dreidimensional aus
			\end{itemize}
			$c_{\mathrm{Luft}} \approx 340 \frac{m}{s} (\mathrm{bei}\ T = 20^{\circ}\mathrm{C})$\\
			$c_{\mathrm{Luft}} \approx 332 \frac{m}{s} (\mathrm{bei}\ T = 0^{\circ}\mathrm{C})$\\
			$c_{\mathrm{Wasser}} \approx 1485 \frac{m}{s} (\mathrm{bei}\ T = 20^{\circ}\mathrm{C})$\\
			$c_{\mathrm{Wasser}} \approx 1404 \frac{m}{s} (\mathrm{bei}\ T = 0^{\circ}\mathrm{C})$
		\subsection{Dopplereffekt}
			\subsubsection{Feststehender Beobachter, bewegliche Schallquelle}
				Annäherung: \\
				$f' = \frac{c}{c-v}\cdot f$\\
				Entfernung: \\
				$f'' = \frac{c}{c+v}\cdot f$
			\subsubsection{Beweglicher Beobachter, feststehende Schallquelle}
				Annäherung: \\
				$f' = \frac{c+v}{c}\cdot f$\\
				Entfernung: \\
				$f'' = \frac{c-v}{c}\cdot f$
		\subsection{Huygensches Prinzip}
			Jeder Punkt einer Wellenfront kann als Ausgangspunkt einer Elementarwelle (Kreis- bzw. Kugelwelle) angesehen werden, die sich mit gleicher Phasengeschwindigkeit und Frequenz wie die ursprüngliche Welle ausbreitet.
		\subsection{Interferenz}
			\begin{itemize}
				\item Überlagerung (gleichfrequenter) Wellen
				\item Auslenkung entsprecht vektorieller Summe der Einzelwellen
			\end{itemize}
			\subsection{Berechnung konstruktive/destruktive Interferenz}
			$s:$ Abstand des betrachteten Punktes zum Erregerzentrum \\
			$\Delta s = |s_1 - s_2|$ (Gangunterschied)\\
			Wenn $\Delta s = n \lambda, n \in \mathbb{N}$: konstruktive Interferenz\\
			Wenn $\Delta s = n \lambda + \frac{\lambda}{2} = (n+\frac{1}{2}) \lambda, n \in \mathbb{N}$: destruktive Interferenz\\\\
			Phasenunterschied zweier Wellen mit Gangunterschied $\Delta s$:\\
			$\phi = \frac{\Delta s}{\lambda} \cdot 2 \pi$
\end{document}